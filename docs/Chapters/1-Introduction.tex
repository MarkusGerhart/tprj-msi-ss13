\section{Einleitung}

Ein Zitat mit Klammern \citep{sprayUser}.

\subsection{Einführung in Spray}

\subsection{Ziel von „Spray Web“}

Das hier Forschungsarbeit: „Wie könnte man die Sache prinzipiell angehen?“

\subsection{Ecore Metamodell}

\subsection{Graphiti}


\section{Anforderungen}

\subsection{Primitive Shapes}

\subsection{Connections}

\subsection{Compartments}

\subsection{Generalisierbarkeit}


\section{Shapes zeichnen}

\subsection{Toolkits}

\subsection{\dd}

\citep{dd}

\subsection{Shape Factory}

Rekursives Zeichnen der Shapes aus einer Definition

\subsection{Compartments}


\section{Code-Generierung}

\subsection{Shape Definitionen}

\subsection{Definitionen für ein zulässiges Modell}


\section{Validierung und Persistierung}

\subsection{Anwendung der abgeleiteten Modell Regeln}

Regeln abgeleitet aus dem Modell mit \dd abfangen.

\subsection{EMF REST}

\subsection{Ecore.js}

\subsection{Ecore mit Server}


\section{Mögliche Aufgaben}

\subsection{Alternative Zeichentoolkits}

go.js oder digaram.js sind eleganter?
Unsere bisherigen Erkenntnisse einbringen.

\subsection{Pflege Codebasis}

Bisheriges verbessern, stabilisieren und testen.

\subsection{Kollobarationsfähigkeit}

\subsection{Validierung ausbauen}

Gibt es andere Möglichkeiten zu validieren oder zu persistieren?
Reine Client-Validierung, also Standalone im Browser oder doch Server?

\subsection{Spray Integration}

Den bisherigen Code in Spray vernünftige integrieren, mit Tests, sowie
den Dev und User Guide entsprechend erweitern.
Generator der Spray Grammatik übersetzt auf die neue Grammatik von Thomas u.a.
portieren?


\section{Zusammenfassung}

Fazi und Zusammenfassung der Ergebnisse.
