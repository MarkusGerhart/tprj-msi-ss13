\section{Umsetzung}

In diesem Kapitel wird beschrieben, wie die Anforderungen praktisch
umgesetzt wurden. Zunächst wird darauf eingegangen, wie Shapes gezeichnet
werden, danach wie Spray passenden Code für Spray Web generieren kann
und zuguterletzt wie Spray Web Modelle validiert und persistiert.

\subsection{Shapes zeichnen}

Es ist essentiell dass die Basisshapes gezeichnet werden können.
Zudem sollen die Basisshapes die in Kapitel \ref{sec.primitivShapes}
beschrieben sind, auch ineinander verschachtelbar sein.
Da die Shapes auch u.a. Textfelder enthalten können, um z.B. das Shape
zu benennen, ist es nötig dass der Benutzer diese verändern kann auch
innerhalb eines verschachtelten Shapes.
Da die Shapes mit Connections zusammen eine Graphstruktur bilden,
müssen die Shapes über Connections miteinander verbunden werden können.

Es wäre ideal, wenn ein Toolkit existieren würde, welches all diese
Anforderungen erfüllt. Falls ein solches Toolkit nicht existiert,
müsste selbst eines mit den in Kapitel \ref{sec.funcAnforderung}
gestellten Anforderungen gebaut werden.

\subsubsection{Toolkits}

Folgende Toolkits haben wir auf ihre Tauglichkeit überprüft:

\begin{itemize}
  \item d3.js
  \item kinetic.js
  \item \dd
\end{itemize}

TODO Liste erweitern und kurz Toolkit beschreiben.

\subsubsection{\dd}

Erst relativ spät haben wir das \dd Toolkit gefunden.
Es erfüllt quasi alle Anforderungen und ist damit der Gewinner:

TODO Tabelle aus Google einfügen.

\dd~wurde von \citep{dd} entwickelt und steht u.a. unter der GNU General
Public Lizenz (GPL) und ist daher auch rein rechtlich geeignet für
das ebenfalls unter einer Open Source Lizenz stehende Spray-Projekt.

\subsubsection{Shape Factory}

Rekursives Zeichnen der Shapes aus einer Definition

\subsubsection{Compartments}


\subsection{Code-Generierung}

Spray Web ist so aufgebaut, dass es von Spray generierten Code annimmt
und verarbeitet. Da Spray intern um eigene Generator-Implementierungen
erweiterbar ist, ist es auch möglich, dass wir ideal für Spray Web
zugeschnittenen Code erzeugen können.

\subsubsection{Shape Layout Definitionen}

\subsubsection{Spray Logik Definitionen}

Definitionen für ein zulässiges Modell.

\subsection{Validierung und Persistierung}

\subsubsection{Anwendung der abgeleiteten Modell Regeln}

Regeln abgeleitet aus dem Modell mit \dd abfangen.

\subsubsection{EMF REST}

\subsubsection{Ecore.js}

\subsubsection{Ecore mit Server}
