\section{Nächste Ausbaustufen}

Hier werden für ggf. künftige Teamprojekte interessante Aufgaben
im Bezug zu Spray Web vorgestellt und kurz beschrieben.

\paragraph{Zeichentoolkits}

Obwohl \dd~ viele Vorteile bietet wie Open Source Lizenz, ein umfassendes
Framework und eine z.Z. aktive Entwicklung, gibt es auch ärgerliche Aspekte,
wie z.B. fehlende Funktionalitäten oder Bugs oder auch dass die Software
nicht öffenltich von einer Gemeinschaft gepflegt wird, sondern nur von
Andreas Herz.

Gibt es besseren Alternativen? Wäre eine nicht Open Source Lösung denkbar?
Folgende Kandidaten: go.js, diagram.js. Oder gibt es andere potentielle
Frameworks? Oder eine schlanke Eigenentwicklung die genau auf Sprays
Bedürfnisse zugeschnitten ist?

Alternativ dazu, könnte man die Änderungen wieder vernünftig in den regulären
\dd~ Quellcode-Baum einpflegen oder eine Kooperation mit Andreas Herz eingehen,
um das Framework von sich noch stärker an die Bedürfnisse von Spray auszurichten.

\paragraph{Pflege Codebasis}

Die bisherige Codebasis ausbauen, verbessern, stabilisieren und testen bzw.
allgemeine Vergesserungen der Architektur insbesondere auf JavaScript-Seite.
Ein automatisiertes Testsystem\footnote{jasmine.js, vows.js, grunt.js,
phantom.js, testling.com, yeoman.io} aufbauen, sowie eine solide Dokumentation
auf Basis von z.B. Sphinx oder Docco.

Portierung des Codes von \dd~ Version 3.0.0 auf die akuellste Version.
Können bessere Abstraktionen als bisher gefunden werden, um den
Draw2d-Patchcode leichter in neue Versionen zu überführen?
Oder kann man sich den Patchcode gänzlich sparen bzw. komplett von extern
injizieren?

Wie könnte eine neue Funktion wie Zooming eingebaut werden? Am besten so,
dass man sich wie in Google Maps durch das Modell hangeln und zoomen kann.

\paragraph{Standalone}

Wenn das Modell mit z.B. ecore.js als JSON gespeichert werden würde,
könnte der Spray Web Editor vollkommen ohne Server, d.h. auch ohne
Kommunikationsoverhead, auskommen.

Das hat den Vorteil, dass ein solcher Modellierungseditor ganz leicht
in Webseiten integriert werden kann, ohne irgendwelche
Konfigurationsorgien. Ein Webseitenbetreiber müsste auch nur seinen existierenden
Server pflegen und könnte mit diesem auf dem bevorzugten Weg direkt mit
dem Editor kommunizieren. Aber gleichzeitig bleibt der Editor so portabel,
dass er einfach auf ein USB-Stick geladen werden kann und quasi ohne irgendwelche
Hilfsmittel ausgeführt werden kann.

Beim Modellieren mit einem grafischen Editor kann es passieren, dass das
Modell sehr groß wird, es wäre daher sehr sinnvoll wenn das Modell
gut dokumentiert werden kann. Daher könnte ein Spray Web Editor in einem
\emph{Nur-Anzeige-Modus}, der standalone auf einer Dokumentationswebseite
läuft, sehr nützlich sein.
Für Dokumentationszwecke wäre es sicher auch sinnvoll, wenn der Editor
auch nur \emph{Teilmengen des Modells} anzeigen kann.


\paragraph{Kollaborationsfähigkeit}

Den Server so ausbauen, dass echte Kollaboration an einem gleichzeitig
Modell möglich wird.
Dazu können mehrere Benutzer auf dem Server angelegt werden, die einen
Speicherplatz für ihre Modelle haben, aber auch das Modell gleichzeitig
mit einem anderen Benutzer bearbeiten können.
Also quasi ein Google Docs für Modelle!

Die andere Möglichkeit wäre mit der Together.js JavaScript-Bibliothek,
eine Kollaboration zu verwirklichen.

\paragraph{Validierung ausbauen}

Gibt es andere Möglichkeiten zu validieren oder zu persistieren?
Ist es möglich und wäre es sinnvoll den Client komplett standalone
zu machen?
Wie kann man Modelle importieren, exportieren bzw. wie unter
Versionskontrolle stellen?

\paragraph{Spray Integration}

Den bisherigen Code in Spray vernünftig integrieren, mit Tests versehen, sowie
den Developer- bzw. User-Guide entsprechend erweitern.
Ein anderes Teamprojekt hat bereits die Spray-Grammatik verbessert,
hier könnte man den Generator auf die neue Spray-Grammatik portieren.


\section{Zusammenfassung}

Spray ist ein Framework mit dem automatisiert grafische Editoren für quasi
beliebige Domänen produziert werden können. Das bedeutet, dass der Spray Web Editor
eine gewisse Flexibilität mitbringen muss, um anhand eines von
Spray automatisch generierten Regelwerk alle essentiellen Editoreigenschaften
bereitstellen zu können.

Ziel der Projektaufttraggeber war es einen funktionierenden grafischen Spray Web
Editor zu erhalten. Dazu müssen Shapes gemäß vorgegebener Definitionen
gezeichnet werden können.
Der Benutzer des Spray Web Editors kann aus den Shapes und Connections ein Modell
„zusammenklicken“. Dieses Modell wird nach einem vorgegebenen Regelwerk auf
Korrektheit überprüft, d.h. der Benutzer ist nur in der Lage valide Modelle
zu erstellen. Zudem kann das so entstandene Modell sowohl als Zeichenfläche als auch als Ecore-Modell
auf dem Server gespeichert werden.

All diese Anforderungen wurden im Zuge dieses Teamprojekts erfasst und konkretisiert.
Es wurden Nachforschungen angestellt, wie man die gesammelten Anforderungen
umsetzen kann. Dabei hat sich das \dd-Framework auf Basis von JavaScript als
geeignet herausgestellt. Dieses dient nun als Basis für Spray Web.
Es übernimmt hauptsächlich die Aufgabe Shapes und Connections zu zeichnen.

Zudem mussten Generatoren, die aus den Spray Grammatiken JSON Dateien
generieren, programmiert werden. Diese JSON Dateien dienen als Regelwerk,
anhand dessen Spray Web sämtliche Shapes, Connections und die Modellvalidierung
ableitet.

Das Teamprojekt konnte all diese Schritte erfolgreich umsetzen. 
Dazu gehörte das Erforschen der prinzipiellen Herangehensweise an die Materie,
die praktsiche Erstellung eines funktionierenden Prototyps,
sowie die entsprechende Dokumentation nach wissenschaftlichen Maßstäben.
