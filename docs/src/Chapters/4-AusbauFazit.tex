\section{Nächste Ausbaustufen}

Hier werden für ggf. künftige Teamprojekte interessante Aufgaben
im Bezug zu Spray Web vorgestellt und kurz beschrieben.

\paragraph{Zeichentoolkits}

Obwohl \dd~ viele Vorteile bietet wie Open Source Lizenz, ein umfassendes
Framework und eine z.Z. aktive Entwicklung, gibt es auch ärgerliche Aspekte,
wie z.B. fehlende Funktionalitäten oder Bugs oder auch dass die Software
nicht öffentlich von einer Gemeinschaft gepflegt wird, sondern nur von
Andreas Herz.

Gibt es besseren Alternativen? Wäre eine nicht Open Source Lösung denkbar?
Die Frameworks go.js, diagram.js sind mit Sicherheit Alternativen. Oder gibt es andere potentielle
Frameworks? Oder eine schlanke Eigenentwicklung die genau auf Sprays
Bedürfnisse zugeschnitten ist?

Alternativ dazu, könnte man die Änderungen wieder vernünftig in den regulären
\dd~ Quellcode-Baum einpflegen oder eine Kooperation mit Andreas Herz eingehen,
um das Framework von sich noch stärker an die Bedürfnisse von Spray auszurichten.

\paragraph{Pflege Codebasis}

Die bisherige Codebasis ausbauen, verbessern, stabilisieren und testen bzw.
allgemeine Verbesserungen der Architektur insbesondere auf JavaScript-Seite.
Ein automatisiertes Testsystem\footnote{jasmine.js, vows.js, grunt.js,
phantom.js, testling.com, yeoman.io} aufbauen, sowie eine solide Dokumentation
auf Basis von z.B. Sphinx oder Docco.

Portierung des Codes von \dd~ Version 3.0.0 auf die akuellste Version.
Können bessere Abstraktionen als bisher gefunden werden, um den
Draw2d-Patchcode leichter in neue Versionen zu überführen?
Oder kann man sich den Patchcode gänzlich sparen bzw. komplett von extern
injizieren?

Wie könnte eine neue Funktion wie Zooming eingebaut werden? Am besten so,
dass man sich wie in Google Maps durch das Modell hangeln und zoomen kann.

\paragraph{Standalone}

Wenn das Modell mit z.B. ecore.js als JSON gespeichert werden würde,
könnte der Spray Web Editor vollständig ohne Server, d.h. auch ohne
Kommunikation zwischen Server und Client, auskommen.

Das hat den Vorteil, dass ein solcher Modellierungseditor ganz leicht
in Webseiten integriert werden kann, ohne aufwändige
Konfiguration und Installation. Ein Webseitenbetreiber müsste nur seinen existierenden
Server pflegen und könnte mit diesem auf dem bevorzugten Weg direkt mit
dem Editor kommunizieren. Aber gleichzeitig bleibt der Editor so portabel,
dass er einfach auf ein USB-Stick geladen werden und ohne irgendwelche
Hilfsmittel ausgeführt werden kann.

Beim Modellieren mit einem grafischen Editor kann es passieren, dass das
Modell sehr groß wird, es wäre daher sehr sinnvoll wenn das Modell
gut dokumentiert werden kann. Daher könnte ein Spray Web Editor in einem
\emph{Nur-Anzeige-Modus}, der standalone auf einer Dokumentationswebseite
läuft, sehr nützlich sein.
Für Dokumentationszwecke wäre es sicher auch sinnvoll, wenn der Editor
auch nur \emph{Teilmengen des Modells} anzeigen kann.


\paragraph{Kollaborationsfähigkeit}

Eine weitere Ausbaustufe ist es den Server mit Kollaborationsfunktionalität auszubauen.
Damit können mehrere Benutzer parallel auf der gleichen Zeichenfläche arbeiten. Die bisherige 
Implementierung ist für Kollaboration in Grundzügen vorbereitet. Die Websocketkommunikation
sorgt bisher für die Persistierung der Zeichenfläche auf dem Server und der Aktualisierung des Ecore-Modells.

Eine Idee ist es mehrere Benutzer auf dem Server anzulegen, welche einen
Speicherplatz für ihre Modelle haben, aber auch das Modell gleichzeitig
mit anderen Benutzer zu bearbeiten. Dank dem verwendeten Webframework Play und Scala kann
Akka verwendet werden. Mit Akka lassen sich verteilte und nebenläufige Applikationen entwickeln.
Die implementierte Websocketkommunikation lässt sich hervorragend mit Akka verknüpfen.
Die Herausforderung ist es die Synchronizität zwischen den Benutzer zu implementieren.
\dd~ liefert einen Commandstack mit, der für diesen Zweck verwendet werden kann. Commands könnten
zwischen den Benutzer geschickt werden um die Zeichenfläche auf allen Clients zu aktualisieren.
Als Vorbild sei hier Google Docs erwähnt.

Die andere Möglichkeit wäre mit der Together.js JavaScript-Bibliothek,
eine Kollaboration zu verwirklichen.

\paragraph{Validierung ausbauen}

Gibt es andere Möglichkeiten zu validieren oder zu persistieren?
Ist es möglich und wäre es sinnvoll den Client komplett standalone
zu machen?
Wie kann man Modelle importieren, exportieren bzw. wie unter
Versionskontrolle stellen?

\paragraph{Spray Integration}

Den bisherigen Code in Spray vernünftig integrieren, mit Tests versehen, sowie
den Developer- bzw. User-Guide entsprechend erweitern sind weitere Aufgaben.
Ein weiteres Teamprojekt hat bereits die Spray-Grammatik verbessert,
hier könnte man den Generator auf die neue Spray-Grammatik portieren.

\paragraph{Generierung und Integration der Styles}

Die Grammatik der Spray Styles wird nicht vom Generator nach JSON konvertiert. Dieses 
Feature fehlt komplett. Mit Ausnahme von Stricharten (dot, dash-dot, etc.) und Positionierung
der Labels, vertikal sowie horizontal, sind keine Styles implementiert.

\paragraph{Nicht umgesetzte Features}

Einige Features sind nicht umgesetzt worden oder müssen optimiert werden. Die folgende Auflistung soll einen
Überblick verschaffen.

\begin{itemize}
  \item Bedingt durch das \dd~ Framwork können Shapes, die zur Connection gehören, nicht rotiert werden.
  \item Speichern/Laden der Compartments funktioniert experimentell. 
  \item Bessere Synchronisierung zwischen Ecore-Modell und der Zeichenfläche.
  \item Optimiertes Mapping zwischen Elementen des Ecore-Modells und der Zeichenfläche.
  \item Die Persistierung des Zeichenflächenmodells könnte durch Commands gelöst werden. Anstatt die Zeichenfläche
  vollständig zu übertragen, würden sich Commands besser eignen, da diese wesentlich kürzer sind.
  \item Undo/Redo für Label ändern. Dieses Feature wird von \dd~ nicht unterstützt.
  \item Im Petrinetz Beispiel ist ein Shape (Edge/Place) auch ein Anker. Dieser Umstand wurde mit 4 Ankern am Rand des Shapes gelöst.
  \item Labels sind manchmal fehlerhaft bzw. schwierig zu editieren. Das könnte man noch „runder“ machen.
  \item Web2Ecore-Interface muss vom Spray Generator für das entsprechende Projekt erzeugt werden.
  \item Weitere fehlende Features z.B. Rapidbuttons.
\end{itemize}


\section{Zusammenfassung}

Spray ist ein Framework mit dem automatisiert grafische Editoren für quasi
beliebige Domänen produziert werden können. Das bedeutet, dass der Spray Web Editor
eine gewisse Flexibilität mitbringen muss, um anhand eines von
Spray automatisch generierten Regelwerk alle essentiellen Editoreigenschaften
bereitstellen zu können.

Ziel der Projektaufttraggeber war es einen funktionierenden grafischen Spray Web
Editor zu erhalten. Das Modell soll nach vorgegebener Grammatik aus Spray im Web Editor gezeichnet werden.
In einem ersten Schritt wurden Generatoren implementiert welche Spray-Grammatiken
(Beispiele Petrinetz und Heizkesselsystem) via Xtext nach JSON konvertiert, damit
der Webeditor die Grammatik lesen kann. Diese JSON Dateien dienen als Regelwerk,
anhand dessen Spray Web sämtliche Shapes, Connections und die Modellvalidierung
ableitet. Der Benutzer des Spray Web Editors kann aus den Shapes, Connections und Compartments ein Modell
„zusammenklicken“. Dieses Modell wird nach einem vorgegebenen Regelwerk auf
Korrektheit überprüft, d.h. der Benutzer ist nur in der Lage valide Modelle
zu erstellen. Zudem kann das so entstandene Modell sowohl als Zeichenfläche als auch als Ecore-Modell über
das implementierte Web2Ecore-Interface auf dem Server gespeichert werden. Dazu ist eine Webserverkommunikation mit Websockets
realisiert worden. Damit das Ecore-Modell tatsächlich mit der Zeichenfläche konsistent ist, muss jedes Ecore-Element
eindeutig seinem Zeichenflächen-Element zuordenbar sein. Die Zuordnung wird mit einem Mapping des Index realisiert.

All diese Anforderungen wurden im Zuge dieses Teamprojekts erfasst und konkretisiert.
Es wurden Nachforschungen angestellt, wie man die gesammelten Anforderungen
umsetzen kann. Dabei hat sich das \dd-Framework auf Basis von JavaScript als
geeignet herausgestellt. Dieses dient nun als Grundlage für Spray Web.
Es übernimmt hauptsächlich die Aufgabe Shapes, Connections und Compartments zu zeichnen.

Das Teamprojekt konnte all diese Schritte letztendlich erfolgreich umsetzen. 
Dazu gehörte die Einarbeitung in die Thematik, Zusammentragen der Anforderungen,
Recherchearbeit zu Technologien, Überlegungen zur Architektur, die praktische Erstellung eines funktionierenden Prototyps,
sowie die entsprechende Dokumentation nach wissenschaftlichen Maßstäben.
