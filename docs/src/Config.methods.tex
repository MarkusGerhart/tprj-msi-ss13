\newcommand{\autor}{Sven Hodapp, Simon Schneeberger, Thorsten Niehues}
\newcommand{\documenttype}{Teamprojekt Dokumentation}
\newcommand{\thema}{Spray Editoren als RIA}
\newcommand{\subtitle}{
Portierung der generierten Spray Editoren von Eclipse Graphiti
auf Webtechnologie als Rich Internet Application, kurz „Spray Web“
}
\newcommand{\institute}{
  \begin{tabular}{r l}
    Institution: & Hochschule Konstanz \\
    Bereich: & Fakultät Informatik \\
    Betreuer: & Prof. Dr. Marko Boger \\
    & Markus Gerhart \\
    Version: & \today
  \end{tabular}
}
\newcommand{\keywords}{spray, graphiti, eclipse, ecore, model driven,
javascript, html, ria, rich internet application}

\newcommand{\typesetTitle}{
  % Source: http://www.latextemplates.com/template/minimalist-book-title-page
  \begingroup

  \raggedleft
  \vspace*{\baselineskip}

  {\Large \autor}\\[0.167\textheight]
  {\LARGE\bfseries \documenttype}\\[\baselineskip]
  \definecolor{mygreen}{RGB}{66,173,82}
  {\textcolor{mygreen}{\Huge \thema}}\\[\baselineskip]
  {\Large \textit{\subtitle}}\par

  \vfill

  {\large \institute}\par

  \vspace*{3\baselineskip}

  \endgroup
}
  % Choose one Titlepage (must provide \typesetTitle)

\newcommand{\titlepageAreal}{
  \thispagestyle{empty}
  \typesetTitle
  \newpage
  \clearpage
  \setcounter{page}{1}
}
\newcommand{\tableOfContentsAreal}{
  \pagenumbering{roman}
  \tableofcontents
  \newpage
  \pagenumbering{arabic}
}
\newcommand{\bibliographyAreal}{
  \newpage
  \addcontentsline{toc}{section}{Literaturverzeichnis}
  % gerabbrv == [1], [2], ...
  % apalike == [Autor, 1987], ...
  % alphadin == [Au87], ...
  % natdin == DIN 1505 == (Autor 1987), ...
  \bibliographystyle{natdin}
  \bibliography{bibliography}
}


\newcommand{\dd}{Draw2D touch}